Сережа занимается паркуром. Поэтому, иногда, вместе со своими друзьями он бегает по крышам соседних гаражей. 
Мальчики давно живут в этом районе, поэтому они с легкостью могут залесть на любую из крыш. Гаражи расположены на прямой. 
Они пронумерованы числами от 1 до $n$ так, с одного гаража можно перебраться на другой тогда и только тогда, когда модуль
разницы их номеров равен 1. Сережа максималист и всегда идет только вперед, поэтому он очень расстраивается, 
когда приходится перелезать с $i$-го гаража на $(i - 1)$-ый. 
Забравшись на какой-то гараж, мальчики начинают бежать по гаражам до тех пор, 
пока им не захочется поесть. Тогда они измеряют, на какую высоту они поднялись за время бега. Чем выше они поднимутся, 
тем счастливее они пойдут в Макдак. Осчастливьте мальчиков!)
\InputFile
На первой строке расположено число $n$ - количество гаражей.
На второй строке расположены $n$ чисел. $i$-ое число - высота iго гаража в метрах.
Все числа во входном файле - натуральные (n \le 100000, $h_i \le 10^9$).
\OutputFile
Выведите, на какую максимальную высоту могут подняться мальчики.

