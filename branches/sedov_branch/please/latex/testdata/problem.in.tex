% Origin: 20110515, 4th SPbSU Math-Mech School Cup
% Problem author: Ivan Kazmenko
% Text author: Anton Maydell
% Tests author: Anton Maydell

\begin{problem}{Игра <<Инфляция>>}
{inflation.in}{inflation.out}
{2 секунды}{256 мебибайт}{}

Вася с Петей часто играют в <<Монополию>>.
Но всё время играть в одну игру надоедает,
поэтому они стали придумывать свою собственную игру <<Инфляция>>.
Вася решил, что в новой игре он будет банкиром.
Начислять обычные проценты по вкладам Васе показалось скучным занятием.
Поэтому он придумал свою уникальную систему работы со вкладами.
Счёт в васином банке всегда содержит целое число рублей.

Пусть на счёте в некоторый день находилось $n$ рублей.
Если $n$ кратно трём, тогда на следующий день счёт будет
содержать $\frac{n}{3}$ рублей.
В противном случае на следующий день на счёте будет $2 n + 1$ рублей.

Вася не хочет, чтобы его банк обанкротился.
Поэтому он решил написать программу для работы с денежными вкладами.
Для начала нужно промоделировать состояние счёта, на который один раз
положили $n$ рублей, а далее не добавляли и не снимали с него средства.
Тогда, если обозначить cумму на счёте в $k$-й день как $a_k$,
получаем, что $a_0 = n$, а каждое следующее $a_i$ получается из $a_{i - 1}$
по описанным выше правилам.

Поэкспериментировав со вкладами, Вася выяснил, что вычисления иногда
можно сократить.
Ведь если в некоторый день $d$ на счёте будет столько же рублей,
сколько было в какой-то из предыдущих дней $c$, то есть $a_d = a_c$, то,
согласно описанным выше правилам, и в последующие дни ситуация повторится:
$a_{d + 1}$ будет равно $a_{c + 1}$, $a_{d + 2}$ будет равно $a_{c + 2}$
и так далее.

Ваша программа по начальному вкладу $n$ должна вычислить
минимальный номер дня $d$ такой, что $a_d = a_c$ для какого-то дня $c < d$,
или выяснить, что такого $d$ не существует.

\InputFile

Входной файл содержит единственное целое число $n$ ($1 \le n \le 10^9$)
"--- начальный вклад.

\OutputFile

Выведите \t{-1}, если требуемого дня $d$ не существует.
В противном случае выведите минимальное значение $d$.

\Examples

\begin{example}
\exmp{
1
}{
2
}%
\exmp{
3
}{
2
}%
\exmp{
4
}{
4
}%
\exmp{
30
}{
-1
}%
\end{example}

\Explanation

В третьем примере последовательность вкладов будет такой:
$a_0 = 4$, $a_1 = 9$, $a_2 = 3$, $a_3 = 1$, $a_4 = 3$, $\ldots$;
здесь $d = 4$, а $c = 2$.

\end{problem}
