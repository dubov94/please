Дан массив чисел. Затем вводятся числа. Требуется определить, сколько же из них содержатся в массиве.

\InputFile
На первой строке находится единственное число $N$ - количество чисел в массиве ($0 \leqslant N \leqslant 10000$).
На второй строке написано $N$ чисел - собственно сам массив.
На следующей строке написано число запросов - $M$ такое, что $0 \leqslant M \leqslant 10000$.
На последней строке находятся $M$ чисел - числа, которые были введены.
Все числа в одной строке разделены ровно одним пробелом. Нет пробелов в начале и конце каждой строки.
Все введенные числа и числа из массива натуральные, содержат не более 300 знаков и не могут иметь ведущих нулей.

\OutputFile
Выведите единственное число - количество введенных чисел, содержащихся в массиве.
