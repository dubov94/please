В файловую ситему одного суперкомпютера проник вирус, который сломал контроль за правами доступа к файлам. Для каждого файла $N_i$ известно, с какими действиями можно к нему обращаться: 

	---Запись($W$),

	---Чтение($R$), 

	---Запуск($X$). 

Вам требуется восстановить контроль над правами доступа к файлам (ваша программа для каждого запроса должна будет возвращать «\texttt{OK}» если над файлом выполняется допустимая операция, или  же «\texttt{Access denied}», если операция недопустима.


\InputFile
В~первой строке входного файла содержится  число $N$ ($1 \leqslant N \leqslant 10\,000$)	~--- количество файлов содержащихся в данной файловой системе.

В~следующих $N$ строчках содержатся имена файлов и допустимых с ними операций, разделенные пробелами. Длина имени файла не превышает 15 символов.

Далее указано чиcло $M$ ($1 \leqslant M \leqslant 50\,000$) ~--- количество запросов к файлам.

В~последних $M$ строках указан запрос вида «Операция Файл». К одному и тому же файлу может быть применено любое колличество запросов.

\OutputFile
Для каждого из $M$ запросов нужно вывести в отдельной строке «\texttt{Access denied}» или «\texttt{OK}».

