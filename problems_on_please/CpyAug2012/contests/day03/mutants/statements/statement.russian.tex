Уже долгое время в Институте Точной Магии и Оздоровения, на кафедре Физической Интенсивной Терапии Идеальных Парней разводят милых разноцветных зверюшек. 
Для удобства каждый цвет обозначен целым числом. 
В один из прекрасных дней в питомнике случилось чудо: все зверюшки выстроились в ряд в порядке возрастания цветов. 
Пользуясь случаем, лаборанты решили посчитать, сколько зверюшек разных цветов живет в питомнике, и, по закону жанра, попросили вас написать программу, которая поможет им в решении этой нелегкой задачи.


\InputFile
В первой строке находятся не более 100000 упорядоченных по неубыванию неотрицательных 
целых чисел~разделенных пробелами~--- цвета зверюшек. 

Во второй строке записаны $M$ целых неотрицательных чисел. M не более 100000.


\OutputFile
Выходной файл должен содержать $M$ строчек. 
Для каждого запроса из второй строки выведите число зверюшек заданного цвета в питомнике.
