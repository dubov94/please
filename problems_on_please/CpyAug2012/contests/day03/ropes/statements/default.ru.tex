С утра шел дождь, и ничего не предвещало беды. Но к обеду выглянуло солнце, и в лагерь заглянула СЭС. 
Пройдя по всем домикам и корпусам, СЭС вынесла следующий вердикт: бельевые веревки в жилых домиках не удовлетворяют нормам СЭС. 
Как выяснилось, в каждом домике должно быть ровно по одной бельевой веревке, и все веревки должны иметь одинаковую длину.
В лагере имеется $N$ бельевых веревок и $K$ домиков. Чтобы лагерь не закрыли, требуется так нарезать данные веревки, 
чтобы среди получившихся веревочек было $K$ одинаковой длины.
Размер штрафа обратно пропорционален длине бельевых веревок, которые будут развешены в домиках. 
Поэтому начальство лагеря стремиться максимизировать длину этих веревочек.

\InputFile

В первой строке заданы два числа~--- $N$ ($1 \leqslant N \leqslant 10001$) и $K$ ($1 \leqslant K \leqslant 10001$). 
Далее в каждой из последующих $N$ строк записано по одному числу~--- длине очередной бельевой веревки. 
Длина веревки задана в сантиметрах. Все длины лежат в интервале от $1$ сантиметра до $100$ километров включительно.


\OutputFile

В выходной файл следует вывести одно целое число~--- максимальную длину веревочек,
удовлетворяющую условию, в сантиметрах. В случае, если лагерь закроют, выведите $0$.
