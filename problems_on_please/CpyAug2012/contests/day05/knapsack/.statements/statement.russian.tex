%Please contest template file
\documentclass[12pt,a4paper,oneside,twocolumn,landscape]{article}

\usepackage[T2A]{fontenc}
\usepackage[utf8]{inputenc}
\usepackage[english,russian]{babel}
\usepackage[russian,landscape]{olymp}
\usepackage{graphicx}
\DeclareGraphicsRule{*}{mps}{*}{}
\graphicspath{{../statements/}}
\usepackage{amsmath,amssymb,amsthm}
\usepackage{epsfig}
\usepackage{import}
\usepackage{wrapfig}
\usepackage{epigraph}

\contest
{}
{Рюкзак}
{28.08.2012}


\renewcommand{\t}[1]{\ifmmode{\mathtt{#1}}\else{\texttt{#1}}\fi}
\renewcommand{\le}{\leqslant}
\renewcommand{\ge}{\geqslant}
\newcommand{\bs}{\mbox{$\backslash$}}

\begin{document}

\raggedbottom

\begin{problem}{Рюкзак}{knapsack.in}{knapsack.out}{2 секунды}{64 мегабайта}{}
\graphicspath{{.././statements/}}


Найдите максимальный вес золота, который можно унести в рюкзаке вместительностью $S$, если есть $N$ золотых слитков с заданными весами.



\InputFile
В первой строке входного файла запиано одно число~--- $S$ ($1 \leqslant S \leqslant 10\,000$). 

Далее следует $N$ неотрицательных целых чисел ($1 \leqslant N \leqslant 300$), не превосходящих 100\,000~--- веса слитков.

\OutputFile
Выведите искомый максимальный вес.


\Examples

\begin{example}
\exmp{
10
1 4 8
}{9
}%
\exmp{
20
5 7 12 18
}{19
}%
\end{example}


\end{problem}


\end{document}
