

На зарядку сегодня утром пришло $N$ ($1 \leqslant N \leqslant 500$) ЛКШат, они построились в ряд. Разумеется,
ребята ходят в разноцветных футболках. Борис Андреевич, наш многоуважаемый физрук,
заметил, что можно попросить некоторых ребят присесть, и тогда для ребят,
которые останутся стоять, будет выполнено следующее: последовательность цветов
их футболок при перечислении слева направо будет такой же как и последовательность
при перечислении справа налево, то есть будет \emph{палиндромом}.

Например, если на зарядку пришли Маша в зеленой футболке, Паша в желтой, Сережа
в красной и Ваня в зеленой, то можно попросить присесть Пашу, тогда последовательность
цветов будет <<зеленый, красный, зеленый>> как слева направо, так и слева направо. Аналогично можно попросить присесть Сережу (последовательность будет <<зеленый, желтый, 
зеленый>>), Пашу и Сережу одновременно, или одного из троих ребят. Таким образом,
всего есть 7 способов добиться того, чтобы последовательность цветов была палиндромом.  

Помогите Борису Андреевич найти количество способов попросить некоторых ЛКШат
присесть, чтобы последовательность цветов футболок оставшихся стоять была палиндромом.

\InputFile
Единственная строка входного файла содержит целые числа, каждое из которых задает цвет футболки ЛКШонка
и изменяется в пределах от 1 до $10^9$. Разные цвета задаются разными числами, а 
одинаковые~--- одинаковыми.

\OutputFile
Выведите в выходной файл одно число~--- искомое количество способов.

