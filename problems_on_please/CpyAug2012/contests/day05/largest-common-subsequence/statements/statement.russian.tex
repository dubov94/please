

Даны две последовательности. Найдите длину их наибольшей общей подпоследовательности (подпоследовательность~--- это то, что можно получить из данной последовательности вычеркиванием некоторых элементов).

\InputFile
В первой строке входного файла через пробел записаны $N$ членов первой последовательности ($1 \leqslant N \leqslant 1000$)~---
 целых чисел, не превосходящих 10\,000 по модулю.
Во второй строке через пробел записаны $M$ членов второй последовательности ($1 \leqslant M \leqslant 1000$)~--- 
целые числа, не превосходящие 10\,000 по модулю.

\OutputFile
В первую строку выходного файла требуется вывести единственное целое число: длину наибольшей общей подпоследовательности или число 0, если такой не существует.
Во вторую строку выходного файла требуется вывести саму наибольшую общую подпоследовательность, 
через пробел (если подпоследовательностей несколько, выведите любую).

