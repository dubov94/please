%Please contest template file
\documentclass[12pt,a4paper,oneside,twocolumn,landscape]{article}

\usepackage[T2A]{fontenc}
\usepackage[utf8]{inputenc}
\usepackage[english,russian]{babel}
\usepackage[russian,landscape]{olymp}
\usepackage{graphicx}
\DeclareGraphicsRule{*}{mps}{*}{}
\graphicspath{{../statements/}}
\usepackage{amsmath,amssymb,amsthm}
\usepackage{epsfig}
\usepackage{import}
\usepackage{wrapfig}
\usepackage{epigraph}

\contest
{}
{Рюкзак с массами}
{01.08.2012}


\renewcommand{\t}[1]{\ifmmode{\mathtt{#1}}\else{\texttt{#1}}\fi}
\renewcommand{\le}{\leqslant}
\renewcommand{\ge}{\geqslant}
\newcommand{\bs}{\mbox{$\backslash$}}

\begin{document}

\raggedbottom

\begin{problem}{Рюкзак с массами}{knapsack2.in}{knapsack2.out}{1 секунда}{64 мегабайта}{}

Дано $N$ предметов массой $m_1$, $\dots$, $m_N$ и стоимостью $c_1$, $\dots$, $c_N$ 
соответственно.

Ими наполняют рюкзак, который выдерживает вес не более $M$. 
Определите набор предметов, который можно унести в рюкзаке, имеющий наибольшую стоимость.

\InputFile
В первой строке вводится натуральное число M, не превышающее 10000.

Во второй строке вводятся $N$ ($N \le 100$) натуральных чисел $m_i$, не превышающих 100.

В третьей строке вводятся $N$ натуральных чисел $с_i$, не превышающих 100.

\OutputFile
Выведите номера предметов (числа от 1 до $N$), 
которые войдут в рюкзак наибольшей стоимости.


\Example

\begin{example}
\exmp{
6
2 4 1 2
7 2 5 1
}{1
3
4
}%
\end{example}


\end{problem}


\end{document}
