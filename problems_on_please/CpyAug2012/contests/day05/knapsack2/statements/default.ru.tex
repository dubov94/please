Дано $N$ предметов массой $m_1$, $\dots$, $m_N$ и стоимостью $c_1$, $\dots$, $c_N$ 
соответственно.

Ими наполняют рюкзак, который выдерживает вес не более $M$. 
Определите набор предметов, который можно унести в рюкзаке, имеющий наибольшую стоимость.

\InputFile
В первой строке вводится натуральное число M, не превышающее $10\,000$.

Во второй строке вводятся $N$ ($N \le 100$) натуральных чисел $m_i$, не превышающих 100.

В третьей строке вводятся $N$ натуральных чисел $c_i$, не превышающих 100.

\OutputFile
Выведите номера предметов (числа от 1 до $N$), 
которые войдут в рюкзак наибольшей стоимости.

