

У одного из преподавателей параллели С.py в комнате живёт кузнечик, который очень любит прыгать по клетчатой одномерной доске. Длина доски~--- $N$ клеток. К его сожалению он умеет прыгать только на $1$, $2$, \dots, $k$ клеток вперёд. 

Однажды преподавателям стало интересно, сколькими способами кузнечик может допрыгать из первой клетки до последней. Помогите им ответить на этот вопрос.


\InputFile
В первой и единственной строке входного файла записано два целых числа~--- $N$ и $k$ ($1 \leqslant N \leqslant 30, 1 \leqslant k \leqslant 10)$.


\OutputFile
Выведите одно число~--- количество способов, которыми кузнечик может допрыгать из первой клетки до последней.

