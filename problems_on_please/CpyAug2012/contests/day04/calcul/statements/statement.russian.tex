

Имеется калькулятор, который выполняет следующие операции:
	\begin{itemize}
	\item умножить число $X$ на $2$;
	\item умножить число $X$ на $3$;
	\item прибавить к числу $X$ единицу.
	\end{itemize}

Определите, какое наименьшее количество операций требуется, чтобы получить из числа~$1$ число~$N$.

\InputFile
Во входном файле написано натуральное число $N$, не превосходящее $10^6$.

\OutputFile
В первой строке выходного файла выведите минимальное количество операций. Во второй строке выведите числа, последовательно получающиеся при выполнении операций. Первое из них должно быть равно 1, а последнее $N$. Если решений несколько, выведите любое.


