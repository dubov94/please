

Совсем недавно девятиклассник Коля прибыл в Летнюю Кинематографическую Школу. 

Первым делом он решил посетить киностудию. В детском лагере <<Бендеровы поляны>>, где ЛКШ проводится в этом году, проложено множество асфальтированных дорожек и их пересечения образуют перекрестки. <<Бендеровы поляны>>~--- необычный лагерь, и поэтому на всех дорожках и перекрестках нарисована разметка, а так же действуют правила дорожного движения, за нарушение которых полагаются штрафы.

Киностудия расположена на юго-востоке от домика, в котором живет Коля, поэтому школьник решил передвигаться только на восток и на юг. Ему очень хотелось побыстрее добраться до киностудии, и потому он решил не обращать внимания на правила, и переходить перекрестки как ему вздумается. Однако, как настоящий ЛКШонок, Коля должен позаботиться о том, чтобы суммарный размер штрафов за его нарушения был минимален. 
Помогите ему в этом.

В вашем распоряжении карта лагеря, представляющая собой клетчатый прямоугольник $N$ на $M$, в котором на пересечении $i$-ой строки и $j$-ого столбца указан размер штрафа при попадании на этот перекресток.

Корпус, в котором живет Коля находится в северо-западном углу лагеря, а киностудия~--- в юго-восточном. Помогите Коле добраться до места назначения, заплатив минимально возможный штраф.

\InputFile
В первой строке входного файла находятся два натуральных числа $N$ и $M$ ($1 \le N, M \le 500$).

В последующих $N$ строках содержатся по $M$ чисел~--- карта лагеря <<Бендеровы поляны>>.


\OutputFile
Выведите одно целое число~--- минимальный размер штрафа,
который придётся заплатить Коле.

