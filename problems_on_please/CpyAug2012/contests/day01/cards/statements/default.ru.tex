Тасование колоды карт происходит следующим образом. Колода разбивается на несколько частей перегородками,
которые нумеруются по номеру стоящей после неё карты (карты нумеруются с единицы). Затем полученные таким образом блоки карт переставляются в обратном порядке
(при этом внутри каждого блока, порядок карт сохраняется).
Требуется по заданным значениям карт и заданными номерами перегородок,
вывести порядок карт после одного такого тасования.

\InputFile
Во входном файле находятся две строки. В первой строке содержатся значения карт по порядку через пробел. При этом гарантируется, что значение каждой карты
 по модулю не более $10^9$, а количество карт не более $10^6$.
Во второй строке содержатся номера перегородок в порядке возрастания через пробел. Перегородки могут ставиться только между картами. Ни в каком
 промежутке между картами не может находиться более одной перегородки.
\OutputFile
В выходной файл надо вывести одну строку~--- полученную в результате тасовки последовательность значений карт.
