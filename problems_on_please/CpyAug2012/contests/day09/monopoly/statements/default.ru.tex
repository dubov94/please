% Statements
В новом варианте игры ``Монополия'' появилась возможность
объединять несколько предприятий в одно для увеличения 
приносимого ими дохода. При это в игре действуют следующие правила:

\begin{enumerate}
\item За один ход можно объединить ровно два предприятия в одно.
При этом стоимость нового предприятия равна сумме стоимостей
двух предприятий до объединения.
\item За совершение операции по объединению предприятий необходимо
заплатить налог в размере 5\% от стоимости объединяемых предприятий.
\end{enumerate}

Коля уже заработал в игре много денег и теперь хочет объединить все свои
предприятия в одно. Он заметил, что общая сумма уплаченного налога
зависит от того, в каком порядке будут объединяться предприятия.
Например, пусть у Коли есть четыре предприятия стоимостью 10, 11, 12 и 13.
Если Коля сначала объединит предприятия 10 и 11
(это обойдется ему в \$1.05), потом результат~--- с 12 (\$1.65), и затем~---
с 13 (\$2.3), то всего он заплатит \$5.
Если же сначала отдельно объединить 10 и 11 (\$1.05), потом~-- 12 и 13
(\$1.25) и, наконец, объединить два полученных предприятия
(\$2.3), то в итоге он заплатит лишь \$4.6.

Помогите Коле определить минимальную сумму денег, необходимую
для объединения всех его предприятий в одно.
\InputFile
В единственной строке входного файла записано
$N$ натуральных чисел ($2 \leq N \leq 200\,00$),
каждое из которых не превосходит 10000~---
стоимости Колиных предприятий.
\OutputFile
В выходной файл выведите минимальную сумму денег
необходимую для объединения всех Колиных предприятий в одно.
