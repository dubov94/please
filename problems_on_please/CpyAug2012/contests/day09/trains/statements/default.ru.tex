% Statements
На вокзале есть $K$ тупиков, куда прибывают электрички. Этот вокзал является их конечной станцией, поэтому электрички, прибыв, некоторое время стоят на вокзале, а потом отправляются в новый рейс (в ту сторону, откуда прибыли).

Дано расписание движения, в котором указаны события прибытия и отбытия для каждой из электричек в хронологическом порядке. Поскольку вокзал --- конечная станция, то электричка может стоять на нем довольно долго, в частности, электричка, которая прибывает раньше другой, отправляться обратно может значительно позднее.

Тупики пронумерованы числами от 1 до $K$. Когда электричка прибывает, ее ставят в свободный тупик с минимальным номером. 

Напишите программу, которая по данному расписанию для каждой электрички определит номер тупика, куда прибудет эта электричка. 

\InputFile
В первой строке вводится число $K$ --- количество тупиков ($1 \leq K \leq 20000$). Далее следуют строки, описывающие события прибытия/отбытия электричек. Каждая электричка задаётся своей противоположной конечной станцией --- строкой длины не более 15 из латинских букв и знаков подчёркивания. Событие \verb"+city" означает, что прибывает электричка из города \verb"city", событие \verb"-city" --- что эта электричка отправляется обратно. Общее количество электричек, фигурирующих в условии --- не более 20000, для каждой фигурирующей электрички присутствуют оба события.

Считается, что в нулевой момент времени все тупики на вокзале свободны. 
% Input file description
\OutputFile
Выведите по одному числу на каждую электричку --- номер тупика, куда её поставят по прибытии. Если тупиков не достаточно для того, чтобы организовать движение электричек согласно расписанию,  выведите два числа: первое должно равняться 0 (нулю), а второе содержать город первой из электричек, которая не сможет прибыть на вокзал.
% Output file description

