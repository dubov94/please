%Please contest template file
\documentclass[12pt,a4paper,oneside,twocolumn,landscape]{article}

\usepackage[T2A]{fontenc}
\usepackage[utf8]{inputenc}
\usepackage[english,russian]{babel}
\usepackage[russian,landscape]{olymp}
\usepackage{graphicx}
\DeclareGraphicsRule{*}{mps}{*}{}
\graphicspath{{../statements/}}
\usepackage{amsmath,amssymb,amsthm}
\usepackage{epsfig}
\usepackage{import}
\usepackage{wrapfig}
\usepackage{epigraph}

\contest
{Берендеевы Поляны}
{Параллель C.py - день 9}
{7 августа 2012}


\renewcommand{\t}[1]{\ifmmode{\mathtt{#1}}\else{\texttt{#1}}\fi}
\renewcommand{\le}{\leqslant}
\renewcommand{\ge}{\geqslant}
\newcommand{\bs}{\mbox{$\backslash$}}

\begin{document}

\raggedbottom

\begin{problem}{Быстрая сортировка}{qsort.in}{qsort.out}{3 секунды}{64 мегабайта}{A}
\graphicspath{{.././qsort/statements/}}
% Statements
Отсортируйте данную последовательность используя алгоритм быстрой сортировки Хоара.
\InputFile
В единственной строке входного файла содержится последовательность, содержащая не более чем 10000 целых чисел.
% Input file description
\OutputFile
В единственной строке выходного файла выведите последовательность в неубывающем порядке. 
% Output file description


\Example

\begin{example}
\exmp{
4 1 4 8 6 6 5
}{1 4 4 5 6 6 8
}%
\end{example}


\end{problem}

\bigskip\bigskip
\begin{problem}{Монополия}{monopoly.in}{monopoly.out}{20 секунд}{64 мегабайта}{B}
\graphicspath{{.././monopoly/statements/}}
% Statements
В новом варианте игры ``Монополия'' появилась возможность
объединять несколько предприятий в одно для увеличения 
приносимого ими дохода. При это в игре действуют следующие правила:

\begin{enumerate}
\item За один ход можно объединить ровно два предприятия в одно.
При этом стоимость нового предприятия равна сумме стоимостей
двух предприятий до объединения.
\item За совершение операции по объединению предприятий необходимо
заплатить налог в размере 5\% от стоимости объединяемых предприятий.
\end{enumerate}

Коля уже заработал в игре много денег и теперь хочет объединить все свои
предприятия в одно. Он заметил, что общая сумма уплаченного налога
зависит от того, в каком порядке будут объединяться предприятия.
Например, пусть у Коли есть четыре предприятия стоимостью 10, 11, 12 и 13.
Если Коля сначала объединит предприятия 10 и 11
(это обойдется ему в \$1.05), потом результат~--- с 12 (\$1.65), и затем~---
с 13 (\$2.3), то всего он заплатит \$5.
Если же сначала отдельно объединить 10 и 11 (\$1.05), потом~-- 12 и 13
(\$1.25) и, наконец, объединить два полученных предприятия
(\$2.3), то в итоге он заплатит лишь \$4.6.

Помогите Коле определить минимальную сумму денег, необходимую
для объединения всех его предприятий в одно.
\InputFile
В единственной строке входного файла записано
$N$ натуральных чисел ($2 \leq N \leq 200\,00$),
каждое из которых не превосходит 10000~---
стоимости Колиных предприятий.
\OutputFile
В выходной файл выведите минимальную сумму денег
необходимую для объединения всех Колиных предприятий в одно.

\Examples

\begin{example}
\exmp{
10 11 12 13
}{4.60
}%
\exmp{
1 1
}{0.10
}%
\end{example}


\end{problem}

\bigskip\bigskip
\begin{problem}{Электрички}{trains.in}{trains.out}{2 секунды}{64 мегабайта}{C}
\graphicspath{{.././trains/statements/}}
% Statements
На вокзале есть $K$ тупиков, куда прибывают электрички. Этот вокзал является их конечной станцией, поэтому электрички, прибыв, некоторое время стоят на вокзале, а потом отправляются в новый рейс (в ту сторону, откуда прибыли).

Дано расписание движения, в котором указаны события прибытия и отбытия для каждой из электричек в хронологическом порядке. Поскольку вокзал --- конечная станция, то электричка может стоять на нем довольно долго, в частности, электричка, которая прибывает раньше другой, отправляться обратно может значительно позднее.

Тупики пронумерованы числами от 1 до $K$. Когда электричка прибывает, ее ставят в свободный тупик с минимальным номером. 

Напишите программу, которая по данному расписанию для каждой электрички определит номер тупика, куда прибудет эта электричка. 

\InputFile
В первой строке вводится число $K$ --- количество тупиков ($1 \leq K \leq 20000$). Далее следуют строки, описывающие события прибытия/отбытия электричек. Каждая электричка задаётся своей противоположной конечной станцией --- строкой длины не более 15 из латинских букв и знаков подчёркивания. Событие \verb"+city" означает, что прибывает электричка из города \verb"city", событие \verb"-city" --- что эта электричка отправляется обратно. Общее количество электричек, фигурирующих в условии --- не более 20000, для каждой фигурирующей электрички присутствуют оба события.

Считается, что в нулевой момент времени все тупики на вокзале свободны. 
% Input file description
\OutputFile
Выведите по одному числу на каждую электричку --- номер тупика, куда её поставят по прибытии. Если тупиков не достаточно для того, чтобы организовать движение электричек согласно расписанию,  выведите два числа: первое должно равняться 0 (нулю), а второе содержать город первой из электричек, которая не сможет прибыть на вокзал.
% Output file description


\Examples

\begin{example}
\exmp{
3
+bologoe
+moscow
-bologoe
+stpetersburg
-stpetersburg
+samara
+saratov
-moscow
-samara
-saratov
}{bologoe 1
moscow 2
stpetersburg 1
samara 1
saratov 3}%
\exmp{
2
+kostroma
+sudislavl
+newvasyuki
-sudislavl
-kostroma
-newvasyuki
}{0 newvasyuki
}%
\end{example}


\end{problem}

\bigskip\bigskip
\begin{problem}{Очередь с приоритетами}{priority-queue.in}{priority-queue.out}{2 секунды}{64 мегабайта}{D}
\graphicspath{{.././priority-queue/statements/}}
% Statements
Реализуйте структуру данных ``очередь с приоритетами'', поддерживающую
следующие операции:

\begin{enumerate}
\item Добавление элемента в очередь.
\item Удаление из очереди элемента с набольшим приоритетом.
\item Изменение приоритета для произвольного элемента, находящегося в
очереди.
\end{enumerate}

\InputFile
Программа получает на вход последовательность команд, по одной команде
в каждой строке. Общее число команд не превосходит
$30\,000$. Команда может иметь один из следующих форматов:

\verb"ADD id priority"~--- добавить в очередь новый элемент
с идентификатором \verb"id" и приоритетом \verb"priority".
Гарантируется, что в очереди нет элемента с таким идентификатором.

\verb"POP"~--- удалить из очереди элемент с наибольшим значением
приоритета. Если таких элементов несколько, то удаляется один
(любой) из них. Гарантируется, что очередь не пуста.

\verb"CHANGE id new_priority"~--- изменить значение приоритета
элемента с идентификатором \verb"id" на значение \verb"new_priority".
Гарантируется, что в очереди есть элемент с таким идентификатором.

Идентификаторы элементов~--- строки, состоящие из строчных латинских
букв длиной не более 10 символов. Идентификаторы~--- произвольные
целые числа.

В самом начале очередь пуста.

\OutputFile
Для каждой команды типа \verb"POP" выведите
идентификатор удаленного элемента и, через пробел,
значение его приоритета.

\Example

\begin{example}
\exmp{
ADD one 1
ADD two 2
ADD three 3
POP
CHANGE one 5
POP
POP
}{three 3
one 5
two 2
}%
\end{example}


\end{problem}


\end{document}
