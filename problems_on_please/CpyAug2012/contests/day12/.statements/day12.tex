%Please contest template file
\documentclass[12pt,a4paper,oneside,twocolumn,landscape]{article}

\usepackage[T2A]{fontenc}
\usepackage[utf8]{inputenc}
\usepackage[english,russian]{babel}
\usepackage[russian,landscape]{olymp}
\usepackage{graphicx}
\DeclareGraphicsRule{*}{mps}{*}{}
\graphicspath{{../statements/}}
\usepackage{amsmath,amssymb,amsthm}
\usepackage{epsfig}
\usepackage{import}
\usepackage{wrapfig}
\usepackage{epigraph}

\contest
{Берендеевы Поляны}
{Параллель C.py - день 12}
{12 августа 2012}


\renewcommand{\t}[1]{\ifmmode{\mathtt{#1}}\else{\texttt{#1}}\fi}
\renewcommand{\le}{\leqslant}
\renewcommand{\ge}{\geqslant}
\newcommand{\bs}{\mbox{$\backslash$}}

\begin{document}

\raggedbottom

\begin{problem}{Дейкстра}{dijkstra.in}{dijkstra.out}{4 секунды}{64 мегабайта}{A}



Дан ориентированный взвешенный граф.

Найдите кратчайшее расстояние от одной заданной вершины до другой.

\InputFile
В первой строке входного файла три числа: $N$, $S$ и $F$ (${1 \leqslant N \leqslant 1000}, {1 \leqslant S, F \leqslant N}$), где $N$~--- количество вершин графа, $S$~--- начальная вершина, а $F$~--- конечная. 
В следующих $N$ строках по $N$ чисел~--- матрица смежности графа, где
$-1$ означает отсутствие ребра между вершинами, а любое целое неотрицательное число, не превосходящее $10\,000$~--- присутствие ребра данного веса. На
главной диагонали матрицы всегда нули.

\OutputFile
Вывести искомое расстояние или $-1$, если пути не существует.


\Example

\begin{example}
\exmp{
3 1 2
0 -1 2
3 0 -1
-1 4 0
}{6}%
\end{example}


\end{problem}

\bigskip\bigskip
\begin{problem}{Кратчайший путь}{distance.in}{distance.out}{3 секунды}{64 мегабайта}{B}



Дан неориентированный взвешенный граф.

Найти кратчайший путь между двумя данными вершинами.

\InputFile
Первая строка входного файла содержит натуральные числа $N$ и $M$ ({$N \leqslant 2\,000$}, $M \leqslant 50\,000$)~---
количество вершин и ребер графа.
Вторая строка входного файла содержит натуральные числа $S$ и $F$ ($1 \leqslant S$, $F \leqslant N$, $S \ne F$)~---
номера вершин, длину пути между которыми требуется найти
Следующие $M$ строк по три натуральных числа $b_i$, $e_i$ и $w_i$~--- номера 
концов $i$-ого ребра и его вес соответственно ($1 \leqslant b_i, e_i \leqslant n$, $0 \leqslant w_i \leqslant 100\,000$).

\OutputFile
Первая строка должна содержать одно натуральное число~--- 
длина минимального пути между вершинами $S$ и $F$. Во второй строке через пробел выведите вершины на кратчайшем пути из $S$ в $F$ в порядке обхода.
Если путь из $S$ в $F$ не существует, выведите $-1$.



\Example

\begin{example}
\exmp{
4 4
1 3
1 2 1
2 3 2
3 4 5
4 1 4
}{3
1 2 3 }%
\end{example}


\end{problem}

\bigskip\bigskip
\begin{problem}{Флойд}{floyd.in}{floyd.out}{2 секунды}{64 мегабайта}{C}



Полный ориентированный взвешенный граф задан матрицей смежности. Постройте матрицу кратчайших путей между его вершинами. 

Гарантируется, что в графе нет циклов отрицательного веса.

\InputFile
В первой строке вводится единственное число $N$ ($1 \leqslant N \leqslant 100$)~--- количество вершин графа. 
В следующих $N$ строках по $N$ чисел задается матрица смежности графа ($j$-ое 
число в $i$-ой строке~--- вес ребра из вершины $i$ в вершину $j$). 
Все числа по модулю не превышают 100. На главной диагонали матрицы~--- всегда нули.

\OutputFile
Выведите $N$ строк по $N$ чисел~--- матрицу расстояний между парами вершин, где
$j$-ое число в $i$-ой строке равно весу кратчайшего пути из вершины $i$ в $j$.



\Example

\begin{example}
\exmp{
4
0 5 9 100
100 0 2 8
100 100 0 7
4 100 100 0
}{0 5 7 13 
12 0 2 8 
11 16 0 7 
4 9 11 0 
}%
\end{example}


\end{problem}

\bigskip\bigskip
\begin{problem}{Цикл отрицательного веса}{negcycle.in}{negcycle.out}{3 секунды}{64 мегабайта}{D}



Дан ориентированный граф. Определите, есть ли в нем цикл отрицательного веса, и если да, то выведите его.

\InputFile
Во входном файле в первой строке число $N$ ($1 \leqslant N \leqslant 80$)~--- количество вершин графа.
В следующих $N$ строках находится по $N$ чисел~--- матрица смежности графа. Все веса ребер
не превышают по модулю $10\,000$. Если ребра нет, то соответствующее число равно $100\,000$.

\OutputFile
В первой строке выходного файла выведите <<\t{YES}>>, если цикл существует или <<\t{NO}>>  в противном случае.
При его наличии выведите во второй строке количество вершин в искомом цикле и в третьей строке~--- вершины,
входящие в этот цикл в порядке обхода.


\Example

\begin{example}
\exmp{
2
0 -1
-1 0
}{YES
2
2 1 }%
\end{example}


\end{problem}

\bigskip\bigskip
\begin{problem}{Заправки}{gasstation.in}{gasstation.out}{2 секунды}{64 мегабайта}{E}

В стране $N$~городов, некоторые из которых соединены между собой дорогами. 
Для того, чтобы проехать по одной дороге, требуется один бак бензина. 
В каждом городе бак бензина имеет разную стоимость. Вам требуется добраться 
из первого города в $N$-й, потратив как можно меньшее количество денег.



Дополнительно имеется канистра для бензина, куда входит столько же бензина,
сколько входит в бак. В каждом городе можно заправить бак, залить 
бензин в канистру, залить и туда и туда, или же перелить бензин 
из канистры в бак.

\InputFile
В первой строке входного файла записано $N$~чисел ($1\le N\le 25$), 
$i$-ое из которых задает стоимость бензина в $i$-ом городе 
(все числа целые в диапазоне от~0 до~100). 

Во следующих строках описаны все дороги (по одной в строке).
Каждая дорога задается двумя числами~--— номерами городов, 
которые она соединяет. Все дороги двухсторонние, между двумя городами
существует не более одной дороги, не существует дорог, ведущих 
из города в себя.

\OutputFile
В выходной файл выведите одно число~--— суммарную стоимость маршрута 
или $-1$, если добраться до нужного города невозможно.

\Example

\begin{example}
\exmp{
1 10 2 15
1 2
1 3
4 2
4 3
}{2
}%
\end{example}


\end{problem}


\end{document}
