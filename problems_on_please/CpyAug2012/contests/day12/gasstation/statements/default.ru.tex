В стране $N$~городов, некоторые из которых соединены между собой дорогами. 
Для того, чтобы проехать по одной дороге, требуется один бак бензина. 
В каждом городе бак бензина имеет разную стоимость. Вам требуется добраться 
из первого города в $N$-й, потратив как можно меньшее количество денег.



Дополнительно имеется канистра для бензина, куда входит столько же бензина,
сколько входит в бак. В каждом городе можно заправить бак, залить 
бензин в канистру, залить и туда и туда, или же перелить бензин 
из канистры в бак.

\InputFile
В первой строке входного файла записано $N$~чисел ($1\le N\le 25$), 
$i$-ое из которых задает стоимость бензина в $i$-ом городе 
(все числа целые в диапазоне от~0 до~100). 

Во следующих строках описаны все дороги (по одной в строке).
Каждая дорога задается двумя числами~--— номерами городов, 
которые она соединяет. Все дороги двухсторонние, между двумя городами
существует не более одной дороги, не существует дорог, ведущих 
из города в себя.

\OutputFile
В выходной файл выведите одно число~--— суммарную стоимость маршрута 
или $-1$, если добраться до нужного города невозможно.
