%Please contest template file
\documentclass[12pt,a4paper,oneside,twocolumn,landscape]{article}

\usepackage[T2A]{fontenc}
\usepackage[utf8]{inputenc}
\usepackage[english,russian]{babel}
\usepackage[russian,landscape]{olymp}
\usepackage{graphicx}
\DeclareGraphicsRule{*}{mps}{*}{}
\graphicspath{{../statements/}}
\usepackage{amsmath,amssymb,amsthm}
\usepackage{epsfig}
\usepackage{import}
\usepackage{wrapfig}
\usepackage{epigraph}

\contest
{Берендеевы Поляны}
{Параллель C.py - день 10}
{8 августа 2012}


\renewcommand{\t}[1]{\ifmmode{\mathtt{#1}}\else{\texttt{#1}}\fi}
\renewcommand{\le}{\leqslant}
\renewcommand{\ge}{\geqslant}
\newcommand{\bs}{\mbox{$\backslash$}}

\begin{document}

\raggedbottom

\begin{problem}{Компоненты связности}{components1.in}{components1.out}{1 секунда}{64 мегабайта}{A}



Дан неориентированный невзвешенный граф. Необходимо посчитать количество его компонент связности.

\InputFile
В первой строке входного файла содержится одно натуральное число $N$ ($N \leqslant 100$)~--- количество вершин в графе. Далее в $N$ строках по $N$ чисел~--- матрица смежности графа: в $i$-й строке на $j$-м месте стоит <<\texttt{1}>>, если вершины $i$ и $j$ соединены ребром, и <<\texttt{0}>>, если ребра между ними нет. На главной диагонали матрицы стоят нули. Матрица симметрична относительно главной диагонали.

\OutputFile
Вывести одно целое число~--- искомое количество компонент связности графа.



\Example

\begin{example}
\exmp{
6
0 1 1 0 0 0
1 0 1 0 0 0
1 1 0 0 0 0
0 0 0 0 1 0
0 0 0 1 0 0
0 0 0 0 0 0
}{3
}%
\end{example}


\end{problem}

\bigskip\bigskip
\begin{problem}{Лесопосадки}{baobab.in}{baobab.out}{1 секунда}{64 мегабайта}{B}



Дан неориентированный невзвешенный граф. Необходимо
определить, является ли он деревом.

\InputFile
В первой строке входного файла содержится одно
натуральное число $N$ ($N \leqslant 100$)~--- количество вершин в
графе. Далее в $N$ строках по $N$ чисел~--- матрица смежности
графа: в $i$-ой строке на $j$-ом месте стоит 1, если вершины
$i$ и $j$ соединены ребром, и 0, если ребра между ними
нет. На главной диагонали матрицы стоят нули. Матрица
симметрична относительно главной диагонали.

\OutputFile
Вывести <<\t{YES}>>, если граф является деревом, <<\t{NO}>> иначе.



\Examples

\begin{example}
\exmp{
6
0 1 1 0 0 0
1 0 1 0 0 0
1 1 0 0 0 0
0 0 0 0 1 0
0 0 0 1 0 0
0 0 0 0 0 0
}{NO}%
\exmp{
3
0 1 0
1 0 1
0 1 0
}{YES}%
\end{example}


\end{problem}

\bigskip\bigskip
\begin{problem}{Долой списывание!}{bipartite.in}{bipartite.out}{1 секунда}{64 мегабайта}{C}

Во время контрольной работы профессор заметил,
что некоторые студенты обмениваются записками.
Сначала он хотел поставить им всем двойки,
но в тот день профессор был добрым,
а потому решил разделить студентов на две группы:
списывающих и дающих списывать, и поставить двойки только первым.

У профессора записаны все пары студентов, обменявшихся записками.
Требуется определить, сможет ли он разделить студентов на две группы так,
чтобы любой обмен записками осуществлялся от студента одной группы
студенту другой группы.

\InputFile
В первой строке входного файла записаны два числа $N$ и $M$~---
количество студентов и количество пар студентов, обменивающихся записками
($1\le N\le 100$, $0\le M\le \frac{N(N-1)}{2}$).
Далее в $M$ строках расположены описания пар студентов:
два числа, соответствующие номерам студентов, обменивающихся записками
(нумерация студентов идёт с 1). Каждая пара студентов перечислена
не более одного раза.

\OutputFile
еобходимо вывести ответ на задачу профессора.
Если возможно разделить студентов на две группы~--- выведите \verb"YES",
иначе выведите \verb"NO".

\Examples

\begin{example}
\exmp{
3 2
1 2
2 3
}{YES
}%
\exmp{
3 3
1 2
2 3
1 3
}{NO
}%
\end{example}


\end{problem}

\bigskip\bigskip
\begin{problem}{Компоненты связности - 2}{components2.in}{components2.out}{3 секунды}{64 мегабайта}{D}



Дан неориентированный невзвешенный граф. Необходимо посчитать количество его компонент связности и вывести их.

\InputFile
Во входном файле записано два числа $N$ и $M$ ($0 < N \leqslant 100\,000$), $0 \leqslant M \leqslant 100\,000$). В следующих $M$ строках записаны по два числа $i$ и $j$ 
($1 \leqslant i,j \leqslant N$), которые означают, что вершины $i$ и $j$ соединены ребром.

\OutputFile
В первой строчке выходного файла выведите количество компонент связности. 
Далее выведите сами компоненты связности в следующем формате: в первой строке количество вершин в компоненте, во второй~--- сами вершины в произвольном порядке.


\Example

\begin{example}
\exmp{
6 4
3 1
1 2
5 4
2 3
}{3
3
1 2 3 
2
4 5 
1
6 
}%
\end{example}


\end{problem}

\bigskip\bigskip
\begin{problem}{Поиск цикла}{cycle.in}{cycle.out}{2 секунды}{64 мегабайта}{E}

Дан ориентированный невзвешенный граф без кратных рёбер.
Необходимо определить, есть ли в нём цикл.

\InputFile
В первой строке входного файла находятся два натуральных числа $N$ и $M$ 
($1 \leqslant N \leqslant 100\,000$, $M \leqslant 100\,000$)~---
количества вершин и рёбер в графе соответственно. Далее в $M$ строках
перечислены рёбра графа. Каждое ребро  задаётся парой чисел~---
номерами начальной и конечной вершин.

\OutputFile
Если в графе нет цикла, то вывести <<\t{NO}>>.
Иначе в первой строке выведите слово <<\t{YES}>>,
во второй строке выведите количество вершин в цикле,
в третьей строке выведите через пробел вершины, входящие
в цикл, в порядке их следования.


\Examples

\begin{example}
\exmp{
2 2
1 2
2 1
}{YES
2
1 2 
}%
\exmp{
2 1
1 2
}{NO
}%
\end{example}


\end{problem}

\bigskip\bigskip
\begin{problem}{TopSort}{topsort.in}{topsort.out}{3 секунды}{64 мегабайта}{F}



Дан ориентированный невзвешенный граф. Необходимо его топологически отсортировать.

\InputFile
В первой строке входного файла даны два натуральных числа $N$ и $M$ ($1 \leqslant N \leqslant 10^5, 1 \leqslant M \leqslant 10^5$)~--- количество вершин и рёбер в графе соответственно. Далее в $M$ строках перечислены рёбра графа. Каждое ребро задаётся
парой чисел~--- номерами начальной и конечной вершин соответственно.

\OutputFile
Вывести любую топологическую сортировку графа в виде последовательности номеров вершин. Если граф невозможно топологически отсортировать, требуется вывести $-1$.



\Examples

\begin{example}
\exmp{
6 6
1 2
3 2
4 2
2 5
6 5
4 6
}{4 6 3 1 2 5
}%
\exmp{
3 3
1 2
2 3
3 1
}{-1
}%
\end{example}


\end{problem}


\end{document}
